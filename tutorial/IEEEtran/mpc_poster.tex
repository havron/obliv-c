\documentclass[conference]{IEEEtran}
% correct bad hyphenation here
\hyphenation{op-tical net-works semi-conduc-tor}

\usepackage{hyperref, textcomp, url, cite}

\begin{document}
\title{Poster: Secure Multi-Party Computation as\\ a Tool for Privacy-Preserving Data Analysis}

\author{\IEEEauthorblockN{Samuel Havron
\IEEEauthorblockA{
\emph{University of Virginia}\\
\href{mailto:havron@virginia.edu}{havron@virginia.edu}\\}
}

}
\maketitle

% IEEE POSTER REQUIREMENTS:
%Submit an abstract no longer than two pages describing the work. The abstract title should begin with the keyword "Poster:". Include all authors with contact information and institutional affiliation in your abstract.
%Your abstract should not exceed the two page limit; non-conforming submissions will not be considered for review.
%If accepted, at least one author must be registered and a final version of the poster abstract submitted by Friday, May 13, 2016.

\begin{abstract}

A \emph{Secure multi-party computation} (MPC) protocol allows two or
more parties to compute a function on sensitive input data provided by
both parties, without revealing anything about the inputs (other than
what can be inferred from the revealed output result).  Social
scientists often work with private datasets that cannot be shared due to
legal restrictions and ownership issues, but many interesting studies
could be enabled if MPC allows joint data analyses on private data
analysis. We are exploring opportunities for using MPC to enable
scientific research that would not otherwise be possible.  We have
developed tools and libraries that enable joint scientific data analyses
with private data, and report on preliminary results using MPC to enable
linear regression analyses over private data.

\end{abstract}

\section{Introduction}

Many social scientists and researchers need to perform statistical
analyses across large, independently-owned datasets for their work, but
they are often met with difficulties in obtaining sensitive data and
computing results in a safe manner.  For instance, an education
researcher may be interested in using statistical methods to analyze the
relationship between family income and grade point average for a
particular school system or collection of school systems.  Obtaining
such sensitive data is often difficult to do without trusting one or
more parties with some of the private input data, as well as considering
ownership and legal restrictions on having clear access to private data.
One government agency has information about incomes, but cannot share it
without violating regulations (and compromising privacy); a local school
district has information about students' grades, but cannot correlate
that with their family incomes.  Secure multi-party computation (MPC) is
a protocol which can be used as a tool for carrying out large-scale
scientific data analysis in these situations without compromising the
privacy of any party's data, or risking it being exposed.

\section{Approach}

Several approaches to MPC have been developed, including
application-specific custom protocols and generic, universal techniques
that can be used to compute any function privately.  We use universal
techniques since it is important that new functions can be developed
quickly and without needing new security proofs, and we anticipate
needing to incorporate auditing and other functionality into the MPC.
The most common universal MPC techniques are based on secret sharing,
homomorphic encryption, and garbled boolean circuits using Yao's
protocol.  We use garbled circuits, which are generally the most
scalable and high performance MPC approach currently known.

Obliv-C (\url{http://oblivc.org}) is a programming language which allows
an application developer to quickly implement scalable, secure MPC
protocols, using the language’s API or writing specific functionality by
extending the language's existing library as well as experimenting with
the implementation of library
protocols\cite{cryptoeprint:2015:1153}. The language is compiled and
built on top of the standard C language, allowing for developers to
integrate C tools and libraries with Obliv-C seamlessly.  Obliv-C
provides an implementation of Yao's garbled circuit protocol for use
with semi-honest adversaries, although it can also be used to implement
other protocols. Using this language to write applications that can
analyze large, privacy-preserving datasets results of building one such
application for linear regression analysis are shown in Preliminary
Results.

The goal of using Obliv-C as a framework for secure computation is to demonstrate 
its performance capabilities and ease of use to developers whom have little 
knowledge of cryptography or circuit structures, but would greatly benefit from 
using secure MPCs to carry out privacy-preserving dataset analysis (perhaps for
social science research). An example of Obliv-C code which calculates and 
reveals the correlation coefficient of a linear regression are seen in Figure~\ref{fig1}.
\begin{figure}[!b]
\centering
\begin{verbatim}
obliv int orsqr = getOblivRSquared();
revealOblivInt(&io -> rsqr,orsqr,0);
\end{verbatim}
\caption{Code snippet showing \emph{obliv} qualifier and reveal API. Revealing an integer
stores the result into a struct which is accessible to specified parties ("0" means
all parties in this context).}
\label{fig1}
\end{figure}
The \emph{obliv} qualifier built into Obliv-C's type system ensures the variable used
is encrypted with the garbled circuit scheme\cite{cryptoeprint:2015:1153}; in this
case, the variable is the correlation coefficient of linear regression analysis,
obtained through some function call to a method within the MPC protocol. The result
is revealed to specified parties through the API \emph{RevealOblivInt()}, which
works by decrypting the \emph{obliv} value into a normal C struct.

\section{Preliminary Results}
\subsection{Scalability}
A linear regression data analysis program was developed to test the scalability 
and speed of Obliv-C as a tool for implementing MPC programs. 
Testing was done using c4.large \emph{Elastic Compute Cloud} (EC2) nodes from \emph{Amazon Web Services} (AWS)\cite{aws:ec2}, which feature 
high frequency Intel Xeon E5-2666 v3 (Haswell) processors optimized specifically for EC2,
two vCPUs, and 3.75 GiB of DRAM.
Two c4.large instances were launched and connected through 
Obliv-C's API for TCP/IP connections via Oblivious Transfer protocol. The instances 
were both located in the same cloud cluster in Oregon; exploring network latency 
impositions with secure MPCs is also an area of interest.

One node instance provided independent ($x$) data points, while the other provided 
dependent ($y$) data points; data points used were 32-bit integers, using fixed-point 
mathematics to convert raw data values into scaled integers, as 
Obliv-C does not currently support floating point numbers.
The time needed to execute the MPC between instances appears to scale linearly 
with the size of the data input; 100K data points finished execution in 12.7 minutes 
on average, 500K completed in 63.7 minutes, and 1 million data points finished execution 
in just over 127 minutes on average. Given the relative cost of using a c4.large instance
(as of writing, \$0.105 per hour), executing this program over larger inputs, such as
10 million data points, will only incur about \$4.45 between two instances in the 
estimated runtime of 21 hours.

Artificial data was generated for testing the scalability of input size, and
considerations to automated data match-ups between two separate datasets were not
implemented; the artifical data was presumed to already be matched and sorted properly.
To provide a clear example of the utility of Obliv-C for analyzing sensitive datasets, 
additional data for computation was obtained from the public New York State Department 
of Health dataset of Hospital Inpatient Discharges from 2011\cite{healthdata:ny}. 
Comparisons between fields such as "Length of Days stayed" and "Total Costs" 
over approximately 2.6 million data points are currently being tested.
This dataset is particularly amenable to analysis, as all data is already 
matched properly and each data value is a comparable number.

\subsection{Discussion}
All numbers are averages over 5 executions between c4.large instances. The datasets used in Preliminary Results assume that data is matched, sorted, and 
comprises of comparable values. Using private set intersection to assist matching 
data with a unique identifier and automatically filtering out incompatible 
data points is considered for future work.

\section{Related Work}
A similar approach in taking distinct federal datasets and comparing them is seen in
Dan Bogdanov \emph{et al.}, where correlations between working hours and failure to
graduate on time in Estonia was investigated, matching over 10 million 
tax records and 500K education records\cite{cryptoeprint:2015:1159}.
This analysis utilized the researchers' own framework for secure
computation, \emph{ShareMind}, a database and analytics system which 
almost exclusively uses three-parties to carry out computations, and uses arithmetic
manipulations to implement secure MPC, rather than boolean circuit evaluation\cite{sharemind}.

Another privacy-preserving approach to analyzing millions of records uses a
combination of homomorphic encryption (for linear computations) and Yao circuits 
(for non-linear computations) in order to compute ridge regression\cite{ridgeregression}.
This approach is designed for many users to send data to a central server called the 
\emph{Evaluator}, in contrast to the primarily two-party model presented in the Obliv-C
language. Yet another approach used for secure multiple linear regression 
relies on protocols based on homomorphic secret sharing, and data which is partitioned 
across several databases\cite{secretsharing}.
 
\section{Conclusion}
Using MPCs for scientific analysis of large datasets is promising for social scientists
and researchers whom would otherwise need to reveal some of one party's input to another
party in order to analyze their data. 
The Obliv-C language is particularly well-suited for implementing such scalable, 
sensitive data analysis between two or more parties, and shows the practicality of 
using secure MPCs for processing large datasets.
Future work invites a closer examination of automatic data-matching between
separate datasets with private set intersection, improving fixed-point integer 
conversion for decimal data values used in computation, and other privacy-preserving
applications.

\begin{thebibliography}{1}

\bibitem{aws:ec2}
  \url{https://aws.amazon.com/ec2/instance-types/}.

\bibitem{healthdata:ny}
  \url{https://health.data.ny.gov/}.

\bibitem{sharemind}
Dan Bogdanov, S. Laur, and J. Willemson,
\emph{Sharemind: A framework for fast privacypreserving
computations}, In 13th European Symposium on Research in Computer
Security (ESORICS), volume 5283 of LNCS, pages 192–206. Springer, 2008.
\url{http://sharemind.cyber.ee}.

\bibitem{cryptoeprint:2015:1159}
    Dan Bogdanov, Liina Kamm, Baldur Kubo, Reimo Rebane, Ville Sokk, Riivo Talviste,
    \emph{Students and Taxes: a Privacy-Preserving Social Study Using Secure Computation},
    Cryptology ePrint Archive, Report 2015/1159,
    2015,
    \url{http://eprint.iacr.org/}.

\bibitem{secretsharing}
  Rob Hall, Stephen E. Fienberg, and Yuval Nardi, 
  \emph{Secure multiple linear regression based on homomorphic encryption}, 
  J. Official Statistics, vol. 27, no. 4, 2011.

\bibitem{ridgeregression}
 Valeria Nikolaenko, Udi Weinsberg, Stratis Ioannidis, Marc Joye, Dan Boneh, Nina Taft,
 \emph{Privacy-Preserving Ridge Regression on Hundreds of Millions of Records},
 in 2013 IEEE Symposium on Security and Privacy (S\&P 2013), pp. 334-348, 
 IEEE Computer Society, 2013,
 \url{http://www.technicolorbayarea.com/papers/2013/NWIJBT13garbled.pdf}.

\bibitem{cryptoeprint:2015:1153}
  Samee Zahur and David Evans, \emph{Obliv-C: A Language for Extensible 
  Data-Oblivious Computation}, Cryptology ePrint Archive, 
  Report 2015/1153, 2015, 
  \url{http://eprint.iacr.org/}.

\end{thebibliography}
\end{document}
